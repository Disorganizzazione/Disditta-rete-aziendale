\documentclass{beamer}
\usepackage[italian]{babel}
\usepackage[utf8]{inputenc}
\usetheme{metropolis}
\graphicspath{{../../Immagini/}}
\def\image[#1][#2]{
  \begin{figure}[H]
  \centering
  \includegraphics[#2]{#1}
  \end{figure}}
\title{Configurazione di una rete aziendale}
\date{\today}
\author{Arianna Masciolini, Claudio Pannacci}
\institute{Università degli Studi di Perugia}
\begin{document}
  \maketitle
  \section{L'azienda}
  	\begin{frame}{Schema fisico}
    	\image[schema_fisico.png][scale=0.30]
  	\end{frame}
  	\begin{frame}{Collegamenti}
  		\begin{itemize}
  			\item \textbf{fibra ottica} per il collegamento degli edifici A, B, C e D
  			\item \textbf{VPN} per l'edificio E
  		\end{itemize}
  	\end{frame}
  \section{Topologia della rete}
  	\begin{frame}{Visione d'insieme}
  		\image[schema_logico.png][scale=0.25]
  	\end{frame}
  	\begin{frame}{Edificio A: Amministrazione}
  		\image[ed_a.png][scale=0.50]
  	\end{frame}
  	\begin{frame}{Edificio B: Bunker}
  		\image[ed_b.png][scale=0.50]
  	\end{frame}
  	\begin{frame}{Edificio C: Cubicoli del Codice}
  		\image[ed_c.png][scale=0.50]
  	\end{frame}
  	\begin{frame}{Edificio D: Dipartimento Demilitarizzato}
  		\image[ed_d.png][scale=0.50]
  	\end{frame}
  	\begin{frame}{Edificio E: Eremo}
  		\image[ed_e.png][scale=0.50]
  	\end{frame}
  \section{Routing}
  	\begin{frame}{Routing e DNS}
  		\begin{itemize}
  			\item \textbf{Routing statico} nelle reti a stella
  			\item Protocollo \textbf{OSPF} nell'area di  backbone
  			\item Due server \textbf{DNS}
  		\end{itemize}
  	\end{frame}
  \section{Misure di sicurezza}
  	\begin{frame}{Misure di sicurezza}
  		\begin{itemize}
  			\item doppio \textbf{firewall} per proteggere la DMZ da accessi interni ed esterni
  			\item server \textbf{proxy}
  			\item \textbf{hardening} server per applicazioni aziendali:
  			\begin{itemize}
  				\item pacchetti TCP filtrati con wrapper
  				\item gestione servizi FTP, SSH e telnet con \texttt{xinetd}
  			\end{itemize}
  			\item \textbf{snmp} per il monitoraggio della rete
  		\end{itemize}
  	\end{frame}
  \section{Costi}
  	\begin{frame}{Preventivo di spesa}
  	\begin{itemize}
  	\item costo componentistica:
  	\end{itemize}
  		\begin{table}[H]
			\centering
			\label{costo componentistica}
			\resizebox{\textwidth}{!}{\begin{tabular}{ll|l|l|}
				\hline
					\multicolumn{1}{|c|}{\textbf{Componente}} & \multicolumn{1}{c|}{\textbf{Quantità}} & \multicolumn{1}{c|}{\textbf{Prezzo cad.}} & \textbf{Prezzo tot.} \\ 						\hline
					\multicolumn{1}{|l|}{Router CISCO 4331 ISR} & 5 & 1100 \$ & 5500 \$ \\ \hline
					\multicolumn{1}{|l|}{Router CISCO ASR 1001} & 1 & 5630 \$ & 5630 \$ \\ \hline
					\multicolumn{1}{|l|}{Modulo fibra} & 4 & 40 \$ & 160 \$ \\ \hline
					\multicolumn{1}{|l|}{Switch CISCO SG300-52p} & 15 & 775 \$ & 11325 \$ \\ \hline
					\multicolumn{1}{|l|}{Switch CISCO SG300-28p} & 4 & 355 \$ & 1420 \$ \\ \hline
					\multicolumn{1}{|l|}{Ubiquiti Networks Unifi WiFi access point} & 1 & 130 \$ & 130 \$ \\ \hline
					\multicolumn{1}{|l|}{Fibra ottica} & 1400 m & 6.34 \$/m & 8876 \$ \\ \hline
					\multicolumn{1}{|l|}{Cavo UTP} & 3000 m & 2.65 \$/m & 7950 \$ \\ \hline
					\multicolumn{1}{|l|}{VPN} &  & 300 \$ annui & 300 \$ annui \\ \hline
 					&  & TOTALE: & \textbf{41291 \$} \\ \cline{3-4} 
 					&  & TOTALE (EUR): & 	\textbf{34323 EUR} \\ \cline{3-4} 
			\end{tabular}}
		\end{table}	
		\begin{itemize}
			\item costo installazione: 20.000 euro.
		\end{itemize}
  	\end{frame}
\end{document}