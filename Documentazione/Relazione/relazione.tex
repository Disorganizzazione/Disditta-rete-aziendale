\documentclass[a4paper,11pt]{article}

\usepackage[utf8]{inputenc}
\usepackage[italian]{babel}
\usepackage{graphicx}
\usepackage{float} %per posizione assoluta delle immagini [H]
\usepackage[colorlinks=true,linkcolor=blue]{hyperref}
\usepackage{nameref} 

\graphicspath{{../../Immagini/}}

\def\image[#1][#2]#3{
  \begin{figure}[H]
  \centering
  \includegraphics[#2]{#1}
  \caption{#3}
  \end{figure}}
\def\boximage[#1][#2]#3{
  \begin{figure}[H]
  \centering
  \fbox{\includegraphics[#2]{#1}}
  \caption{#3}
  \end{figure}}
  
\title{Progettazione e configurazione di una rete aziendale}
\author{Arianna Masciolini, Claudio Pannacci}

\begin{document}

\maketitle
\newpage
\tableofcontents
\newpage
\section{Requisiti e soluzioni proposte}
\boximage[schema_fisico.png][scale=0.5]{Pianta degli edifici della DisDitta.}
Scopo di questo lavoro sono la progettazione e la configurazione di una rete per conto di una azienda che lavora in ambito informatico. 
Gli edifici sono dislocati come mostrato in figura e sono richiesti l'accesso protetto ad internet, copertura WiFi nell'edificio D, un server di posta elettronica, un server web, due server DNS, un server proxy, uno di backup ed un server per applicazioni aziendali, da proteggere con particolare attenzione.
Per quel che riguarda la loro collocazione, si è deciso di collocare una DMZ nell'edificio D, detto pertanto Dipartimento Demilitarizzato, contenente il server di posta, il server web, il server proxy ed uno dei due DNS.
Poiché deve esserci anche copertura WiFi, l'edificio D ospita inoltre il server DHCP.
L'edificio A (Amministrazione) ospita il secondo server DNS, mentre nell'edificio B (Bunker) si trova il server di backup. Il server per applicazioni aziendali si trova nel cuore dell'azienda: l'edificio C (Cubicoli del Codice), dove si trovano gli uffici degli sviluppatori. Infine, l'edificio E (Eremo), sede legale della ditta, non opsita alcun server.
\begin{table}[h]
\centering
\label{riepilogo}
\begin{tabular}{|l|l|l|}
\hline
\multicolumn{1}{|c|}{\textbf{Nome edificio}} & \multicolumn{1}{c|}{\textbf{Numero utenti}} & \multicolumn{1}{c|}{\textbf{Server}} \\ \hline
Amministazione & 100 & DNS \\ \hline
Bunker & 100 & Backup \\ \hline
Cubicoli del Codice & 260 & Appl. aziendali \\ \hline
Dipartimento Demilitarizzato & 240 & \begin{tabular}[c]{@{}l@{}}DHCP, DNS, web, \\ mail, proxy\end{tabular} \\ \hline
Eremo & 50 & - \\ \hline
\end{tabular}
\caption{Riepilogo.}
\end{table}
\newpage
\section{Schema logico della rete}
\boximage[schema_logico.png][scale=0.35]{Schema logico.}
\subsection{Edificio A: Amministrazione}
\subsection{Edificio B: Bunker}
\subsection{Edificio C: Cubicoli del Codice}
\subsection{Edificio D: Dipartimento Demilitarizzato}
\subsection{Edificio E: Eremo}
\newpage
\section{Configurazione delle interfacce di rete} %?
\section{Routing}
\section{Misure di sicurezza}
\subsection{Firewall}
\subsection{Hardening: il server per applicazioni aziendali}
\subsection{Monitoraggio della rete}
\newpage
\section{Preventivo di spesa}
\end{document}
