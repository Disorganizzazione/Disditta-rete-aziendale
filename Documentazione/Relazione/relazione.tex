\documentclass[a4paper,11pt]{article}

\usepackage[utf8]{inputenc}
\usepackage[italian]{babel}
\usepackage{graphicx}
\usepackage{float} %per posizione assoluta delle immagini [H]
\usepackage[colorlinks=true,linkcolor=blue]{hyperref}
\usepackage{nameref} 

\graphicspath{{../../Immagini/}}

\def\image[#1][#2]#3{
  \begin{figure}[H]
  \centering
  \includegraphics[#2]{#1}
  \caption{#3}
  \end{figure}}
\def\boximage[#1][#2]#3{
  \begin{figure}[H]
  \centering
  \fbox{\includegraphics[#2]{#1}}
  \caption{#3}
  \end{figure}}
  
\title{Configurazione di una rete aziendale}
\author{Arianna Masciolini, Claudio Pannacci}

\begin{document}

\maketitle
\newpage
\tableofcontents
\newpage
\section{Requisiti}
%schema fisico
\newpage
\section{Schema logico della rete}
\subsection{Edificio A: Amministrazione}
\subsection{Edificio B: Bunker}
\subsection{Edificio C: Cubicoli del Codice}
\subsection{Edificio D: Dipartimento Demilitarizzato}
\subsection{Edificio E: Eremo}
\newpage
\section{Configurazione delle interfacce di rete} %?
\section{Routing}
\section{Misure di sicurezza}
\subsection{Firewall}
\subsection{Hardening: il server per applicazioni aziendali}
\subsection{Monitoraggio della rete}
\newpage
\section{Preventivo di spesa}
\end{document}